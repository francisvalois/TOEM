%!TEX root = ../rapport.tex
%!TEX encoding = UTF-8 Unicode

% Chapitres "Introduction"

% modifié par Francis Valois, Université Laval
% 31/01/2011 - version 1.0 - Création du document


\label{s:experimentation}
\chapter{Laboratoire 6}
\section{Projet 1: Paramètres théoriques selon les modes}
\subsection{Calcul de la fréquence de $f_c$}
Les modes sont   
6561679790.02625 (1,0)
13123359580.0525 (2,0)
14763779527.5591 (0,1)

\subsection{Calcul de la vitesse de groupe $v_g$ selon la fréquence}

ça commence a 7 ghz
  \begin{table}[htbp]
    \centering
    \begin{tabular}{|c|c|c|c|c|c|c|c|c|c|c|c|c|c|c|c|c|} \hline
	0.0487 & 0.0398 & 0.0340 & 0.0299 & 0.0267 & 0.0243 & 0.0222 & 0.0206 & 0.0191 & 0.0179 & 0.0168 & 0.0159 & 0.0150 & 0.0143 & 0.0136 & 0.0130 & 0.0124\\ 
	0.0000 & 0.0000 & 0.0000 & 0.0000 & 0.0000 & 0.0615 & 0.0413 & 0.0328 & 0.0278 & 0.0244 & 0.0218 & 0.0199 & 0.0183 & 0.0170 & 0.0159 & 0.0149 & 0.0141\\ 
	0.0000 & 0.0000 & 0.0000 & 0.0000 & 0.0000 & 0.0000 & 0.1131 & 0.0486 & 0.0356 & 0.0291 & 0.0251 & 0.0222 & 0.0201 & 0.0184 & 0.0170 & 0.0159 & 0.0149 \\\hline
    
    \end{tabular}%
        \caption{Impédances de charges selon les deux méthodes pour deux différents cas}
    \label{tab:4}%
\end{table}%



\subsection{Calcul de la longueur d'onde $\lambda_g$ selon la fréquence}

\subsection{Calcul de la l'impédance intrinsèque transverse du guide $\eta_{GTE}$ selon la fréquence}