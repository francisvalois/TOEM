%!TEX root = ../rapport.tex
%!TEX encoding = UTF-8 Unicode

% Chapitres "Introduction"

% modifié par Francis Valois, Université Laval
% 31/01/2011 - version 1.0 - Création du document


\label{s:experimentation}
\chapter{Laboratoire 6}
\section{Projet 1: Paramètres théoriques selon les modes}
Selon les documents de spécifications obtenues en ligne, la largeur interne du guide d'onde de type WR-90 est de 22.86 mm et la hauteur interne est de 10.16 mm.

\subsection{Calcul de la fréquence de $f_c$}
La fréquence de coupure pour un guide d'onde rectangulaire en fonction des différents modes TE est donnée par l'équation suivante :

\begin{equation}
	f_{c_{mn}} = \frac{v_p}{2}\sqrt{\left(\frac{m}{a}\right)^2+\left(\frac{n}{b}\right)^2}
\end{equation}
Où:
\begin{itemize}
	\item $v_p$ est la vitesse de propagation de l'onde dans le guide d'ondes, ici la vitesse de la lumière dans le vide soit $3\cdot10^8m/s$;
	\item $m$ et $n$ sont les deux indices du mode TE à évaluer;
	\item $a$ et $b$ sont respectivement la largeur et la hauteur du guide d'onde en mètre, ici 0.02286m et 0.01016m.
\end{itemize}

À l'aide de l'équation précédente, il suffit de trouver pour quels indices TE $(m,n)$, la fréquence de coupure est plus petite que la fréquence du mode d'opération. Dans le cas d'une opération à une fréquence de 15 GHz, nous trouvons les 3 modes d'opérations suivants :
\begin{itemize}
	\item $F_c = 6.56 GHz$ pour un mode d'opération $(1,0)$;
	\item $F_c = 13.12 GHz$ pour un mode d'opération $(2,0)$;
	\item $F_c = 14.76 GHz$ pour un mode d'opération $(0,1)$
\end{itemize}


Les fréquences de coupures pour les modes supérieurs ne sont pas présentées, car elles sont supérieures à la fréquence de fonctionnement, ce qui va atténuer ou empêcher la diffusion de l'onde dans le guide.

\subsection{Calcul de la vitesse de groupe $v_g$ selon la fréquence}
La vitesse de groupe pour chacun des modes est obtenue à l'aide de l'équation suivante :

\begin{equation}
	v_g = v_p\sqrt{1-\left(\frac{f_c}{f}\right)^2}
\end{equation}
Où :
\begin{itemize}
	\item $v_p$ est la vitesse de propagation de l'onde dans le guide d'ondes, ici la vitesse de la lumière dans le vide soit $3\cdot10^8m/s$;
	\item $f_c$ est la fréquence de coupure pour un mode donné;
	\item $f$ est la fréquence de l'onde dans le guide.
\end{itemize}


Ainsi pour une fréquence de fonctionnement allant de 0 à 25 GHz, nous obtenons les vitesses de groupe affichées à la figure \ref{fig1}. Les trois modes de fonctionnement obtenus précédemment sont respectivement représentés par les 3 courbes sur le graphique. De plus, les valeurs numériques ne sont pas présentées, car celles-ci ne font qu'alourdir le rapport et peuvent être déduites de la figure.

\begin{figure}[htbp]
    \centering
    \includegraphics[scale=0.45]{fig2.png}
    \caption{Figure présentant la vitesse de groupe d'une onde en fonction de la fréquence de fonctionnement pour 3 différents modes TE}
    \label{fig1}
\end{figure}

\subsection{Calcul de la longueur d'onde $\lambda_g$ selon la fréquence}
La longueur d'onde dans le guide est donnée par l'équation suivante :

\begin{equation}
	\lambda_g = \frac{\lambda}{\sqrt{1-\left(\frac{f_c}{f}\right)^2}}
\end{equation}
Où:
\begin{itemize}
	\item $\lambda$ est la longueur d'onde du signal traversant le guide; 
	\item $f_c$ est la fréquence de coupure pour un mode donné;
	\item $f$ est la fréquence de l'onde dans le guide.
\end{itemize}


Sachant que la longueur d'onde dans le guide est donnée par l'équation \ref{eq1}, il est possible de trouver $\lambda_g$ pour une fréquence entre 0 et 25 GHz. Les longueurs d'ondes guidées sont affichées à la figure \ref{fig2}. Comme précédemment, les 3 différents modes sont affichés sur la figure et les résultats numériques ne sont pas affichés pour simplifier la présentation du rapport.

\begin{figure}[htbp]
    \centering
    \includegraphics[scale=0.45]{fig1.png}
    \caption{Figure présentant la longueur d'onde guidées en fonction de la fréquence de fonctionnement pour 3 différents modes TE}
    \label{fig2}
\end{figure}

\begin{equation}
	\label{eq1}
	\lambda = \frac{v_p}{f}
\end{equation}

\subsection{Calcul de la l'impédance intrinsèque transverse du guide $\eta_{GTE}$ selon la fréquence}
L'impédance intrinsèque transverse peut être obtenue à l'aide de l'équation suivante :
\begin{equation}
	\eta_{GTE} = \frac{\eta}{\sqrt{1-\left(\frac{f_c}{f}\right)^2}}
\end{equation}
Où:
\begin{itemize}
	\item $\eta$ est l'impédance intrinsèque du guide d'onde, dans notre cas $\eta = 377\Omega$, car nous sommes dans l'air;
	\item $f_c$ est la fréquence de coupure pour un mode donné;
	\item $f$ est la fréquence de l'onde dans le guide.
\end{itemize}

La figure \ref{fig3} représente l'impédance intrinsèque transverse pour les 3 modes d'intérêts entre des fréquences de 0 et 25 GHz.

\begin{figure}[htbp]
    \centering
    \includegraphics[scale=0.45]{fig3.png}
    \caption{Figure présentant l'impédance intrinsèque d'une guide d'onde en fonction de la fréquence de fonctionnement pour 3 différents modes TE}
    \label{fig3}
\end{figure}


\section{Projet 2: Mesure directe de la fréquence}
8V

Fréquence de la diode : 10538 GHZ

Passe de -0.45 à -1.4

10V

passe de -0.25 à -0.90

Fréquence de la diode : 10541 GHZ

4V

Le système est beaucoup plus sensible au variations

\section{Projet 3: Mesure directe de la fréquence}