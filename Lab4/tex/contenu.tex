%!TEX root = ../rapport.tex
%!TEX encoding = UTF-8 Unicode

% Chapitres "Introduction"

% modifié par Francis Valois, Université Laval
% 31/01/2011 - version 1.0 - Création du document


\label{s:experimentation}
\chapter{Laboratoire 2}
\subsection{A)}
L'appareil décrit dans la question est un coupleur directif. Les figures \ref{fig:coupleurdirectif1} et \ref{fig:coupleurdirectif2} représente le coupleur directif fabriqué par Lab-Volt.

\subsection{B)}
Oui, car avec un potentiel nul, la diode de Gunn ne peut tout simplement pas conduire un courant.

\subsection{C)}
Nous modulons par un signal carré de 1kHz car l’électronique autour du “0” Hertz n’est pas facile à cause des tensions de polarisation et de décalage (“offset”) qu’on élimine souvent par des condensateurs de
découplage ou avec des ajustements externes.

\subsection{D)}
En ayant un signal entrant de 2mW et un couplage de 10 dB, nous obtenons un signal C+D de $\frac{2mW}{10} = 200\mu W$. Sachant que le coupleur possède une directivité de 30 dB, nous obtenons les équations suivantes : 

\begin{equation}
	C+D = C+C/1000 = 200\mu W
\end{equation}

En solutionnant, nous trouvons que C = 199.8 $\mu W$ et D = 0.2 $\mu W$.

Si le signal de 2 mW entre dans l'entrée 1, nous obtenons dans l'accès 2 un signal de 1.8 mW, dans l'accès 3, un signal de 199.8 $\mu W$ et dans l'accès 4 un signal de de 0.2 $\mu W$.

\subsection{E)
Il suffit de modifier le calibre vers la plage -20 dB sans modifier le gain, ni avec la fréquence centrale. 

\subsection{F}
En diminuant la largeur de bande, on diminue la puissance du bruit sur le signal total. 


\subsection{}


